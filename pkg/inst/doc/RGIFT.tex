\documentclass[a4paper]{article}

\usepackage{verbatim}

\usepackage[latin1]{inputenc}

\usepackage[authoryear]{natbib}


% \VignetteIndexEntry{An R Package for producing questionnaires in GIFT format}
% \VignetteDepends{RGIFT}
% \VignetteKeyword{GIFT}
% \VignetteKeyword{Moodle}

\title{RGIFT: An R Package for producing questionnaires in GIFT format}
\author{Virgilio G�mez-Rubio, Mar�a Jos� Haro-Delicado and\\ Franciso Parre�o-Torres}

\newcommand{\RR}{\texttt{R}\ }
\newcommand{\code}[1]{\texttt{#1}}
\newcommand{\pkg}[1]{\textsf{#1}}

\usepackage{Sweave}
\begin{document}

\maketitle


\begin{abstract}
This paper introduces \RR package \pkg{RGIFT} to produce questionnaires in the
GIFT format. This is a popular format used by several virtual learning
environments. This package, combined with the \RR programming language, can be
used to produce a number of questionnaires to test students' skills in an easy
way.


\end{abstract}


\section{Introduction}


\citet{ZeileisBettina:2009} describe the development of exams in an automated
way in order to deal with a large number of students. In this work we describe
a similar problem but provide a different solution based on the use
of on-line questionnaires in a Course Management System.

This work has been triggered by the development of the Bolgone Process\footnote{http://www.ond.vlaanderen.be/hogeronderwijs/bologna/}
and the need for a continuous assessment of the students during all their
learning process. Lecturers are required to assess the students' developments
through regular tests.

Most engineering degrees in the University of Castilla-La Mancha (UCLM)
have a first-year course on Statistics. In order to conduct a continuous
evaluation, which is at the heart of the Bologne Process, a number of tests are
taken by the students in regular intervals. In particular, a test is taken at
the end of each practical session in the computer lab. 

The total number of first-year students attending these courses is between 300
and 400 in the Degrees of Computer Science and Industrial Engineering where we
teach. UCLM provides all students and academic staff access to a Course
Management System \citet{Moodle}. Moodle is a complex system which allows
lecturers to upload files with course notes, excercises and other interesting
options to assess the students.

In particular, Moodle provides a suitable framework to develop tasks for the
students. The most basic one is to provide a place to upload files so that
lectures can grade them. Another interesting option includes on-line
questionnaires.  In this case, the student is presented with a number of
different types of questions that he has to answer in a limited time. Moodle
provides a grafical user interface to create these questions for this can be
slow and, sometimes, cumbersome if a large number of questions need to be
developed. Needless to say that once the student has finisehd and submitted his
answers, he may get his mark immediately.

For this reason we have developed package \code{RGIFT}. With it a large
number of differnt types of questions can be generated and used to create
questionnaires for the students. In this paper we describe not only 
the different options in this package but also how to take advantage of
the Moodle platform to produce questionnaires to assess students.

\section{Creating questionnaires with GIFT}

The \citet{GIFT} format provides a simple but flexible way of producing
questionnaires. Moodle can import and export questionnaires in GIFT format and
there are several tools to convert from GIFT into other popular formats for
on-line questionnaires. Questions must be written using UTF-8 text encoding.

Package \pkg{RGIFT} implements several functions to print questions in the GIFT
format. The output can be redirected to a file using, for example, the
\code{sink()} function. Similarly, \pkg{Sweave} \citep{Leisch:2002} can be used
to embed the \RR code so that the ouput is directly included in the R output
file.

The GIFT formats can deal with different types of questions whose formats are
described in the following sections.  Comments can be added to the
questionnaire text file by using function \code{GIFTcomment()}. Also, questions
can be grouped into categories by using function \code{GIFTcategory()}. Once
this function is called, all question will be included in the same section
until another category is set.


\subsection{Types of questions}

\subsubsection{Multiple Choice Questions}

Multiple Choice questions are those for which several possible answers are
given and one must be chosen. Full mark is obtained only when the right answer
is chosen and the student can get a penalty (i.e., negative mark) if a wrong
answer is chosen rather than leaving the answer empty.  Other variants include
the option of having alternative "less right" answers which will not give the
full mark. This can be controlled by assigning different weights to right and
wrong answers.

The following \RR code shows how to produce a 3-question questionnaire
using Multiple Choice Questions:


\verbatiminput{MCQExample.R}
\begin{Schunk}
\begin{Sinput}
> library(RGIFT)
> source("MCQExample.R")
\end{Sinput}
\end{Schunk}

Question 1 provides a very simple question about the mean of a set of numbers.
In addition to the question, a vector of possible answers is given and the
index of the right answer is available in \code{rightans}. Also, a negative
weight is assigned to all the wrong answers (\code{wwrong="-33.333" }).
Full marks will have a weight of 100.00 and secondary right answers
can be given a lower positive weight.
Question 2 is a similar answer where the right answer is the first one in the
vector of possible answers (this is the default).  

Question 3 is an example of a Multiple Choice question where there is a right
answer which gives full mark, another not so good answer (which gives a 50\%)
and wrong answers (with a penalty).  As one of the answers provides the full
mark, Moodle only allows a single answer to be chosen.

Finally, Question 4 is an example of a trully Multiple Choice question where
multiple answers must be chosen. Note that this is so because no single answer
provides a full mark and that in order to get a 100\% two answers must be
chosen.

\subsubsection{True-False Questions}

True-False questions are those in which a statement is made and the student has
to mark it either as true or false. The following code shows some examples
using function \code{GIFTTF()}:

\verbatiminput{TFQExample.R}
\begin{Schunk}
\begin{Sinput}
> source("TFQExample.R")
\end{Sinput}
\end{Schunk}


\subsubsection{Short Answer Questions}

Short Answer questions are those for which the students has to write down the 
answer. This can be produced using function \code{GIFTSA()} as follows:

\verbatiminput{SAQExample.R}
\begin{Schunk}
\begin{Sinput}
> source("SAQExample.R")
\end{Sinput}
\end{Schunk}

Similarly to the Multiple Choice questions, different weights for right and
worngs answers can be set. Multiple answers are possible as well.

\subsubsection{Matching Questions}

Matching Questions provide two groups of words and/or sentences that need
to be matched. This can be produced as follows using function \code{GIFTM()}:

\verbatiminput{MQExample.R}
\begin{Schunk}
\begin{Sinput}
> source("MQExample.R")
\end{Sinput}
\end{Schunk}

\subsubsection{Missing Word Questions}

In Missing Words questions the students need to fill in an empty space
in a sentence with a word from a list. These can be defined
using function \code{GIFTMW()}:


\verbatiminput{MWQExample.R}
\begin{Schunk}
\begin{Sinput}
> source("MWQExample.R")
\end{Sinput}
\end{Schunk}



\subsubsection{Numerical Questions}

With  numerical questions students are asked to compute something and write the
numeric answer down. Rather than matching the exact number it is better to
allow for some error around the exact value so that valid answers are those
found in an short interval. This questions are defined as following using
function \code{GIFTNQ()}:


\verbatiminput{NQExample.R}
\begin{Schunk}
\begin{Sinput}
> source("NQExample.R")
\end{Sinput}
\end{Schunk}


In principle, the range for valid answers is the answer plus/minus the error
(as in Question 1 above).  If the answer is a vector of length two, then that
is the range of valid answers (as in Question 2 above).

\subsubsection{Essays}

Essays are simply a question followed by a text box where the student is asked
to write some text. This can be done, for example, to ask students to write
some code to perform a particular task.  The answers can be downloaded later
and run through \RR to test the students' code automatically. This questions can
be created with command \code{GIFTE()}:

\verbatiminput{EExample.R}
\begin{Schunk}
\begin{Sinput}
> source("EExample.R")
\end{Sinput}
\end{Schunk}


\subsubsection{Description}

Finally, Descriptions provide some text with no answer required. These
descriptions are useful to include hints or tips in the questionnaires
or to mark the beginning of different blocks of questions. Descriptions
can be added with command \code{GIFTD()}:

\verbatiminput{DExample.R}
\begin{Schunk}
\begin{Sinput}
> source("DExample.R")
\end{Sinput}
\end{Schunk}

\subsection{Formatting text}

In principle, questions and answers will be shown in plain text.
However, other text formats can be used with Moodle.
A full description of them is available at 
\verb+http://docs.moodle.org/21/en/Formatting_text+.
The text format is specified before the question text inside square
brakets. The options are:

\begin{description}

\item[moodle] is for Moodle Auto-Format text. This is used in 
text boxes in Moodle (for example, to send e-mails , add descriptions, etc.).

\item[html] is for HTML text. This will allow the use of most HTML tags to
customise the text, add external links, etc.

\item[plain] is for plain text. The question will be shown as it is written
and special tags (for example, in HTML) will be shown as they are written with
no other effect.

\item[markdown] is for Markdown text. This is another simple format to enhance
the text. A complete description of this format is available at 

\verb+http://docs.moodle.org/22/en/Markdown+

\end{description}


In addition, it is necessary to take care of a number of special
characters (such as \, \$, \{, \} and others) as they may not be displayed
correctly. Function \code{RGIFTparse()}
will look for these specials characters in a string and substitute them
by an equivalent expression which will not interfere with the GIFT format.
The following example will take a string with some of these special
characters and return a valid one which can be passed to any of the
functions to produce the questions:

\begin{Schunk}
\begin{Sinput}
> GIFTparse("Take care with $, {, } and ~.")
\end{Sinput}
\begin{Soutput}
[1] "Take care with \\$, \\{, \\} and \\~."
\end{Soutput}
\end{Schunk}
\noindent
The double backslash is need because text is printed using the \code{cat()}
function but a single backslash will be printed.


Finally, as text must be encoded in UTF-8 function \code{iconv()} can be used
to convert  text strings from the local character encoding into UTF-8.
A simple example in Spanish could be:

\verbatiminput{UTF8.R}
\begin{Schunk}
\begin{Sinput}
> source(file("UTF8.R", encoding="ISO-8859-1"))
\end{Sinput}
\end{Schunk}
\noindent
Note that the previous text has been written in an environment with an
ISO-8859-15 encoding. A good discussion on the use of different encodings
in \RR can be found in the manual page of function \code{Encoding()}.



\subsection{Including images}

Questions sometimes require the use of images that display data or summary
results. In principle, these images should be among the files uploaded to
Moodle.  Images can be uploaded to the course site in Moodle in a ZIP file and
then uncompressed there.  Alternatively, the images can be created and stored
in the same server but outside Moodle (for example, the lecturer's web site) so
that they are linked from the question text using HTML tags.

A convenient approach is to define an image repository and create all images in
JPEG or PNG format. Then the images can be uploaded to the image repository and
the specfific link to the image included in the question text. A simple example
would look as follows:

\begin{Schunk}
\begin{Sinput}
> #Image repository
> repos<-"http://www.uclm.es/profesorado/vgomez/imgtests/"
> #Create image
> imgfile<-"q1.png"
> png(imgfile, width=480, height=240)
> par(mfrow=c(1,4))
> #Sample several random variables and plot histogram
> hist(rbinom(100, 10, .25), main="(a)", xlab="x")
> hist(rnorm(100, 10*.25, sqrt(10*.25*.75)), main="(b)", xlab="x")
> hist(rexp(100, 1/(10*.25)), main="(c)", xlab="x")
> hist(runif(100, 8, 12), main="(d)", xlab="x")
> dev.off()
> #Create question in HTML format
> qtxt<-paste("[html]What\'s the histogram for a Binomial",
+    "variable with n=10 and prob=0.25?\n<p>\n<img src=\'", 
+    repos, imgfile, "\'>\n<p>", sep="")
> #Print question
> sink(file="imgqtion.txt", type="output")
> GIFTMC(qtxt, c("(a)", "(b)", "(c)", "(d)"), rightans=1, wwrong="-33.333")
> sink()
> 
\end{Sinput}
\end{Schunk}
\noindent
Note that different images must have different names but that this can
be easily implemented in \RR.



\subsection{Dealing with equations}

In some cases equations need to be displayed in the question text or in the answers.
Moodle supports many different ways of including equations (see 
\verb+http://docs.moodle.org/22/en/Mathematics+). However, when producing
questions, it is necessary to include the equations within the question text.
Moodle allows the use of in-line \LaTeX\ syntax for equations provided that
these are included using \LaTeX\ math-mode (i.e., using the two-dollar signs
before and after the equation). For example:


\begin{Schunk}
\begin{Sinput}
> sink(file="eqqtion.txt")
> qtxt<-paste("[html]The expectation of a random variable which is \n",
+    "Binomial with $$n=10$$ and $$\\pi=0.7$$ is ", sep="")
> GIFTMC(qtxt, c("0.7", "1", "7", "0.07"), rightans=3)
\end{Sinput}
\begin{Soutput}
[html]The expectation of a random variable which is 
Binomial with $$n=10$$ and $$\pi=0.7$$ is 
{
~%0%0.7
~%0%1
=7
~%0%0.07
}
\end{Soutput}
\begin{Sinput}
> sink()
\end{Sinput}
\end{Schunk}




\section{Design and use of questionnaires in Moodle}

The previous examples can be used to create a set of different questions which
will possibly belong to different categories. The use of categories is
encouraged since it allows questions to be grouped into different topics.
Moodle allows some customization on the group of questions. 

For example, a questionnaire can be made of, say, 15 questions taken at random
from different categories. The number of questions from each category can be
set as well, so that all students will have the same number of questions (taken
at random) from each category.  This is helpful to design questionnaires that
include questions about different sections in the syllabus.

Regarding the output given to the student, the right answers and final score
can be shown once the questionnaire has been finished and the answers
submitted. However, in certain cases it may be better not to show the right
answers immediately and wait until the end of the exam (to make sure that all
students have finished).  

Although setting up a Moodle environment may seem complex (as it will require a
web server and other software), a virtual appliance could be used instead. This
will provide a pre-installed system will all the required software. We have
tried TurnkeyLinux (\verb+http://www.turnkeylinux.org/moodle+) to start a
Moodle server in a few minutes. Among other features, this includes LDAP
access so that external users can be authenticated and students can login
using their university username and password. Hence, lecturers can set up
a complete Moodle system in the local network.

\section{Real Example}


We have used \pkg{RGIFT} to produce a number of tests for the practicals of a
first-year course in Statistics in the degrees in Computer Science and
Industrial Engineering in the campus of Albacete (University of Castilla-La
Mancha). These practicals are organised in seven two-hour sessions. Sessions
one to six are used to train the students in the uses of \RR for descriptive
and inferential statistics. In these sessions the last 30 minutes are used to
answer a questionnaire about the contents of the practical. These
questionnaires are made of 15 different questions which are randomly selected
from the pool of question for that practical. Each group of questions is
assigned to a category depending on the practical and, sometimes, the topic
(for example, linear regression or ANOVA).
The students where allowed to repeat the test up to three times and the highest
score was kept. The students where given the right answers and mark
after they completed each test.

The last session is an overall exam of the course practicals. Here, the
students have two hours to complete the questionnaire. As this exam needs to
cover all topics, the categories where used to design a questionnaire where two
questions where taken (at random) from the questions in previous practicals.
The students were only given the total mark when they submitted the exam and
the actual list of right ans wrong answers was available after all the students
had taken the exams a few days later.

In all cases, our aim was to provide each student with a similar (but
different) list of questions so that all of them faced tests of similar
complexity but in a way that prevented cheating.
The code to produce these questions is available in files 
\verb+test_exam-en.R+ (in English) and \verb+test_exam-es.R+ (in Spanish)
in directory \verb+RGIFT/doc/exams+ and can be easily view using, for example,
\code{vi(file=system.file("doc/exams/test\_exam-es.R", package="RGIFT"))}.

Also, we provided similar tests on the theoretical parts of the course so that
they could take the tests at home and have immediate feedback on his knowledge
of the topics in the course. Moodle keeps a record of the students' activities.
In particular, for questionnaires the complete list of answer, socres, etc.  is
available so that the lecturers can provide a complete analysis of the
responses. This is of interest in order to find what parts of the syllabus the
students have not understood well or particularly difficult questions.



\section{Conclusions}

In this paper we have introduced package \pkg{RGIFT} which provides a 
convenient and easy way of producing different types of questions
for on-line questionnaires. This can be fully imported into the
Course Management System Moodle in order to develop on-line assessment,
self study materials and on-line courses. 

\section{Acknowledgements}

This work is part of project ``Matem�TICas con software libre'' funded by
Consejer�a de Educaci�n y Ciencia, Junta de Comunidades de Castilla-La Mancha
for 2010-2012.


%\bibliographystyle{spbasic}
\bibliographystyle{plainnat}
%\bibliography{RS_handbook_SpatStat_INLA}
\bibliography{RGIFT}

Carrera, E. (2002). Teaching statistics in secondary school. An overview: From the curriculum to reality. In B. Philips (Ed.), Proceedings of the Sixth International Conference on  Teaching of Statistics. Cape Town: IASE. CD ROM.

Gal, I. (2002).  Adult?s statistical literacy. Meanings, components, responsabilities. International Statistical Review, 70(1), 1-25


%\appendix
%
%\section{Code for practicals}
%
%\verbatiminput{test_exam-en.R}



\end{document}
